\documentclass[11pt,a4paper,english]{article}
\usepackage[T1]{fontenc}
\usepackage[utf8]{inputenc}
\usepackage{babel}
\usepackage[svgnames]{xcolor}
\usepackage{listings}
\usepackage{lmodern}
\usepackage{graphicx}
\lstset{language=c++,
	numberstyle=\ttfamily,
	basicstyle=\ttfamily,
	keywordstyle=\color{blue}\ttfamily,
	stringstyle=\color{red}\ttfamily,
	commentstyle=\color{gray}\ttfamily,
	morecomment=[l][\color{magenta}]{\#},
	breaklines=true,
  postbreak=\mbox{\textcolor{red}{$\hookrightarrow$}\space},
}

\title{Pengolahan Citra Video}
\author{Azzam Wildan Maulana}
\date{16 April 2022}

\begin{document}
\maketitle
\begin{figure}[h]
\centering
\includegraphics[scale=1]{its.png}
\end{figure}	

\pagebreak

\section{Pengenalan program}
Pada tugas PCV kali ini yaitu membuat sebuah virtual-backgroud dari sebuah video. Pada tugas ini saya menggunakan OpenCV C++ karena saya ingin belajar hal yang lebih baru. Program yang saya buat memiliki dua sub-program, yang pertama yaitu sub-program untuk melakukan treshold untuk background yang akan digunakan, dan kedua adalah menampilkan hasilnya.   
Program yang saya buat memanfaatkan lib-yaml untuk melakukan dynamic load-save nilai konfigurasi untuk tresholdnya. Dan juga, menggunakan fitur trackbar untuk mengatur nilai treshold HSV nys.     

\section{Penjelasan program}
\begin{enumerate}
  \item Memuat video dari sebuah source video dan sebuah vbg dari sebuah gambar
  \item Melakukan resampling pada setiap frame nya
  \item Mencari nilai treshold HSV untuk backgroundnya
  \item Membuat mask dan inverted mask-nya dari nilai treshold tersebut 
  \item Meng-burn mask ke frame video
  \item Menggabungkan hasil dari burn mask menjadi satu frame
  \item Menampilkan frame final yang dibuat pada step 6
\end{enumerate}

\section{Source code}
Adapun source code yang telah saya buat adalah sebagai berikut: 
\lstinputlisting[label={foo},caption={main.cpp}, language={C++}]{main.cpp}

\section{Dokumentasi}

\end{document}
